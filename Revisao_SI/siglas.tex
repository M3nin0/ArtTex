%O documentclass abaixo foi uma ajuda do Alexandre
\documentclass[
	12pt,				% tamanho da fonte
	%openright,			% capítulos começam em pág ímpar (insere página vazia caso preciso)
	%twoside,			% para impressão em recto e verso. Oposto a oneside
	openany,			%Para nao pular folhas quando um paragrafo novo começa. Oposto de Twoside e openright
	a4paper,			% tamanho do papel.
	chapter=TITLE,		% títulos de capítulos convertidos em letras maiúsculas
	section=TITLE,		% títulos de seções convertidos em letras maiúsculas
	%subsection=TITLE,	% títulos de subseções convertidos em letras maiúsculas
	%subsubsection=TITLE,% títulos de subsubseções convertidos em letras maiúsculas
	english,
	brazil				% o último idioma é o principal do documento
]{abntex2}
\usepackage[brazil]{babel}
\usepackage[utf8]{inputenc} %Pacote de linguas
\usepackage[normalem]{ulem}
\usepackage[T1]{fontenc}
\usepackage{lipsum}
\usepackage{cmap}
\usepackage{graphicx}
\usepackage[brazilian,hyperpageref]{backref}
\usepackage[alf]{abntex2cite} % Citações padrão ABNT
\usepackage{rotating}
\usepackage{float}
%%%%%%%%%%%%%%%%%%%%%%%%%%%%%%%%%%
%Comando da image By Filipe

\newcommand{\imagem}[3]{
	\begin{figure}[htb]
		\begin{center}
			\includegraphics[scale=0.5]{#1}
		\end{center}
		\caption{#2}%\label{#3}
	\end{figure}
}

%%%%%%%%%%%%%%%%%%%%%%%%%%%%%%%%%%


%%%%%%%%%%%%%%%%%%%%%%%%%%%%%%%%%%
\autor{2 semestre} % (\\) --> Pula linha

\setlength{\parskip}{0.6cm}
% Identação e espaçamento entre letras
\setlength{\parindent}{1.3cm}
\frenchspacing

% Adicionando idioma
\selectlanguage{brazil}

\titulo{Sistemas de informação\\Revisão}
\data{Abril, 2017}
\tipotrabalho{Trabalho Acadêmico}


\begin{document}

\imprimircapa

\imprimirfolhaderosto

\tableofcontents

\maketitle

\newpage

\chapter{Siglas}

Neste documento irei listas cada sigra encontrada durante as aulas de S.I, para que possamos compreender melhor a estrutura do conteúdo passado pelo professor.

\begin{itemize}

\item ERP -Enterprise Resource Planning
	\begin{itemize}
		\item[] O ERP é um sistema que integra todos os dados e processos da empresa em um único sistema. Esta integração é total, ele únifica todos os processos da empresa, facilitando o acesso a informação.
	\end{itemize}

\item SIGE - Sistemas Integrado de Gestão Empresarial
	\begin{itemize}
		\item[] O Sige tem o mesmo significado que ERP, porém com as siglas em português.
	\end{itemize}


%http://www.ideacrm.com.br/marketing-de-relacionamento-b2b-afinal-o-que-%C3%A9-prm/
\item PRM - Partner Relationship Management
	\begin{itemize}
		\item[] É um sistema que auxilia na relação com parceiros, este tem por objetivo facilitar a comunicação entre as empresas e seus parceiros.
	\end{itemize}


\item Data mining
	\begin{itemize}
		\item[] É o processo de explorar grandes quantidades de dados à procura de padrões consistentes, como regras de associação ou sequências temporais
	\end{itemize}


%http://www.administradores.com.br/artigos/marketing/crm-o-que-e-crm-e-como-funciona/34063/
\item CRM - Customer Relationship Management
	\begin{itemize}
		\item[] De acordo com a Gartner o CRM é uma estratégia de negócio voltada ao entendimento e antecipação das necessidades e potenciais de uma empresa.
		\item[] É um sistema utilizado para o contato com cliente e a facilitação dos canais de comunicação entre a empresa e seu cliente.
	\end{itemize}

\item ERM - Enterprise Risk Management
	\begin{itemize}
		\item[] Sistema que ajuda a empresa identificar, analisar, avaliar, monitorar e gerenciar seus riscos corporativos utilizando uma abordagem integrada.
		\item[] A solução reúne todos os dados relacionados à gestão de riscos em um único ambiente, incluindo uma biblioteca reutilizável de riscos e seus respectivos controles e avaliações, eventos, tais como perdas e não conformidades, indicadores de desempenho e planos de tratamento.
	\end{itemize}

\item SAF - Sales Force Automation
	\begin{itemize}
		\item[] Sistema que ajuda as equipes de vendas a gerir todos os seus processos.
	\end{itemize}

\item EIS - Sistema de Informação Executiva
	\begin{itemize}
		\item[] Sistema criado com objetivo de oferecer facilidades para administração de grandes empresas.
	\end{itemize}


\item OLAP - Processo Analítico On-line
	\begin{itemize}
		\item[] Processo que realiza a ánalise de grandes quantias de dados, com diferentes perspectivas.
	\end{itemize}

\item DSS - Sistema de Suporte a Decisão
	\begin{itemize}
		\item[] Sistemas de apoio à decisão é uma classe de Sistemas de Informação. Refere-se simplesmente a um modelo genérico de tomada de decisão que analisa um grande número de variáveis para que seja possível o posicionamento a uma determinada questão.
	\end{itemize}

\item OLTP - Processamento On-line de Transação 
	\begin{itemize}
		\item[] Um tipo de sistema projetado para suportar aplicativos orientados à transações. Os sistemas OLTP são projetados para responder imediatamente às solicitações do usuário, e cada solicitação é considerada uma única transação. As solicitações podem envolver adicionar, recuperar, atualizar ou remover dados.
	\end{itemize}


\item EDP - Processamento Eletrônico de Dados
	\begin{itemize}
		\item[] Sigla para denominar que certas processos de dados são realizados pelo computador.
	\end{itemize}


\newpage
As siglas demonstradas a seguir, são siglas utilizadas para denominar as estratégias de comunicação entre a empresa e seu cliente/parceiros.

\item B2B - Business to Business
	\begin{itemize}
		\item[] São as formas de comunicação de empresa para empresa 
	\end{itemize}

\item B2C - Business to Consumer
	\begin{itemize}
		\item[] Formato de comunicação entre a empresa o consumidor
	\end{itemize}

\item C2C - Consumer to Consumer
	\begin{itemize}
		\item[] Comunicação de consumidor para consumidor
	\end{itemize}

\item B2G - Business to Government
	\begin{itemize}
		\item[] Empresa para governos
	\end{itemize}

\item B2E - Business-to-Employee
	\begin{itemize}
		\item[] Empresa para funcionário
	\end{itemize}

Todas essas siglas de B2X são utilizadas para denominar as multiplas faces da empresa para atrair investidores, parceiros e clientes para seu negócio.

\end{itemize}
\end{document}