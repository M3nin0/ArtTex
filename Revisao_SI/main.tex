%O documentclass abaixo foi uma ajuda do Alexandre
\documentclass[
	12pt,				% tamanho da fonte
	%openright,			% capítulos começam em pág ímpar (insere página vazia caso preciso)
	%twoside,			% para impressão em recto e verso. Oposto a oneside
	openany,			%Para nao pular folhas quando um paragrafo novo começa. Oposto de Twoside e openright
	a4paper,			% tamanho do papel.
	chapter=TITLE,		% títulos de capítulos convertidos em letras maiúsculas
	section=TITLE,		% títulos de seções convertidos em letras maiúsculas
	%subsection=TITLE,	% títulos de subseções convertidos em letras maiúsculas
	%subsubsection=TITLE,% títulos de subsubseções convertidos em letras maiúsculas
	english,
	brazil				% o último idioma é o principal do documento
]{abntex2}
\usepackage[brazil]{babel}
\usepackage[utf8]{inputenc} %Pacote de linguas
\usepackage[normalem]{ulem}
\usepackage[T1]{fontenc}
\usepackage{lipsum}
\usepackage{cmap}
\usepackage{graphicx}
\usepackage[brazilian,hyperpageref]{backref}
\usepackage[alf]{abntex2cite} % Citações padrão ABNT
\usepackage{rotating}
\usepackage{float}
%%%%%%%%%%%%%%%%%%%%%%%%%%%%%%%%%%
%Comando da image By Filipe

\newcommand{\imagem}[3]{
	\begin{figure}[htb]
		\begin{center}
			\includegraphics[scale=0.5]{#1}
		\end{center}
		\caption{#2}%\label{#3}
	\end{figure}
}

%%%%%%%%%%%%%%%%%%%%%%%%%%%%%%%%%%


%%%%%%%%%%%%%%%%%%%%%%%%%%%%%%%%%%
\autor{2 semestre} % (\\) --> Pula linha

\setlength{\parskip}{0.6cm}
% Identação e espaçamento entre letras
\setlength{\parindent}{1.3cm}
\frenchspacing

% Adicionando idioma
\selectlanguage{brazil}

\titulo{Sistemas de informação\\Revisão}
\data{Abril, 2017}
\tipotrabalho{Trabalho Acadêmico}


\begin{document}

\imprimircapa

\imprimirfolhaderosto

\tableofcontents

\maketitle

\newpage

\chapter{Introdução}

%Será feito por último no caso desta matéria

\chapter{Sistemas de informação (SI)}


Os sistemas de informações tem mudado drasticamente o mundo dos negócios e das pessoas, mas o que são sistemas e sistemas de informação ? 

\textbf{Sistema}
	'Sistema pode ser definido como um conjunto de elementos/componentes que mantêm relações entre si. Os componentes e as relações formam as características específicas do sistema. Para o conjunto de relações entre componentes se associa uma ação/dinâmica e resultados'

\textbf{Sistema de informação}
	'Um Sistema de Informação (SI) é um sistema cujo elemento principal é a informação. Seu objetivo é armazenar, tratar e fornecer informações de tal modo a apoiar as funções ou processos de uma organização'

Um sistema de informação é composto de um sub-sistema social e outro sub-sistema automatizado, onde o social trata das pessoas e dos processos, e o automatizado sobre os computadores e redes de telecomunicação que interligam os elementos sociais.

\textbf{Atividades com informações que contém sistemas de informação}
	\begin{itemize}
		\item Entrada;
		\item Processamento;
		\item Saída;
		\item \textit{Feedback};
		\item A ação de clientes e acionsitas também interage com o sistema de informação da empresa.
	\end{itemize}

\textbf{Os sitemas de informação vão muito aĺém dos computadores}

Os sistemas de informação conseguem abrangir muito mais do que apenas computadores, eles conseguem unificar a organização, os processos e pessoas, e os computadores para que consiga oferecer solução para os mais diversos problemas da organização.
Eles se divedem nos tópicos de:
	\begin{itemize}
		\item Organizacional: Encontrasse processos, as formas hierarquicas da empresa, e quando a empresa utiliza os sitemas de informação, a cultura organizacional desta reflete nos sitemas de informação. Veja a \textit{United Parcel Service} (UPS) que se preocupa muito com o cliente e a forma de atendimento;
		\item Pessoa: As pessoas são quem desenvolvem, mantém e utilizam o sistema de informação, a ação de cada pessoa sobe os sistemas de informação podem afeta-lo de forma positiva ou negativa;
		\item Tecnologia: São as plataformas ais quais os sistemas de informação são construidos sobre. Sendo \textit{Software}, \textit{Hardware}, gerenciamento de dados, redes e telecomunicação.
	\end{itemize}

\subsection{O papel dos sistemas de informação no ambiente de negócios contemporâneo}

Cada vez mais os sistemas de informação vem tomando conta dos ambientes corporativos, já é extremamente comum ver empresas com sites e blogs para postar conteúdo com relação a seus serviços e desta forma disseminar sua marca. O \textit{marketing} digital já se tornou algo poderoso e é muito utilizado por todas as empresas. Para perceber isso basta entender que o famoso \textit{spam} foi modificado, estruturado e tornado uma grande ferramentas de \textit{marketing} com os \textit{e-mail marketing}. Até mesmos leis foram criadas para assegurar que dados e informações se tornassem algo a se preocupar por parte das empresas que oferecem serviços \textit{on-line}.

\subsubsection{As novas tendências dos sistemas de informação}

%https://antonioricardo.org/2013/03/28/o-que-e-saas-iaas-e-paas-em-cloud-computing-conceitos-basicos

Como dito anteriormente, os SI crescem a cada dia, é possível perceber isso com o avanço de técnologias como:
	\begin{itemize}
		\item Computação em nuvem;
		\item Software como serviço (SaaS - \textit{Software as a service});
		\item Infraestrutura como serviço (IaaS - \textit{Infraestruct as a service});
		\item Plataforma como serviço (PaaS - \textit{Plataform as a service});
		\item Trabalho remoto amplamente utilizado;
		\item Cada vez mais os serviços remotos diminuem os valores das operações.
	\end{itemize}

A evolução das técnologias trouxeram também novas formas de negócios, hoje existem trabalhos pare resolver problemas que a 5 anos atrás não existiam. Com esta evolução as empresas perceberam que seria necessário investir em tecnologia, para se manterem competitivas e assim se fez, hoje as empresas investem em T.I para conseguir:
\begin{itemize}
	\item Excelência operacional
		\begin{itemize}
			\item A excelência operacional gera mais lucros, veja que há empresas que são gigantes por terem técnologias que as auxiliam nisto, como exemplo a Amazon, empresa de varejo gigantesca que oferece os mais variados serviços. E com sistema de informação consegue realizar entregas de seus pedidos em até alguns minutos depois que foram feitos.
		\end{itemize}
	\item Novos produtos, serviços e modelos de negócios
		\begin{itemize}
			\item As novas técnologias permitem a criação de novas formas de negócios, como citado anteriormente, veja a indústria fonográfica que a alguns anos atrás era gigantesca e hoje perde cada vez mais espaço para empresas como o \textit{Spotify} que permitem que qualquer artista publique suas obras sem terem grandes fortunas.
		\end{itemize}
	\item Relacionamento mais estreito com clientes e fornecedores
		\begin{itemize}
			\item Com a melhora e velocidade dos atendimento subindo, os clientes sempre retornam.
		\end{itemize}
	\item Melhor tomada de decisão
		\begin{itemize}
			\item Sistemas como o \textit{Business Intelligence} ou BI conseguem interpretar as informações de todas as operações realizadas pela empresa, e emitir relatórios precisos sobre toda a empresa, permitindo uma tomada de decisão muito melhor.
		\end{itemize}
	\item Vantagem competitiva
		\begin{itemize}
			\item Dependendo do ramo em que a empresa atua, a técnologia pode fazer com que o produto oferecido pode ser superior ao concorrente e ainda mais barrato.
		\end{itemize}
	\item Sobrevivência
		\begin{itemize}
			\item Mesmo se a empresa não trabalha com técnologia ela será obrigada comprar técnologia para se manter ativa em seu ramo.
		\end{itemize}
\end{itemize}


Assim é possível perceber que as empresas investem em técnologia para ter:
	\begin{itemize}
		\item Vantagens competitivas,
		\item Melhores serviços,
		\item Menos erros,
		\item Maior precisão, 
		\item Produtos de melhor qualidade,
		\item Aperfeiçoamento,
		\item Melhor eficiência,
		\item Maior produtividade, 
		\item Maiores oportunidades,
		\item Administração mais eficiente,
		\item Automatização de tarefas rotineiras,
		\item Custos reduzidos,
		\item Maior e melhor controle sobre as operações,
		\item Melhores tomadas de decisões..
	\end{itemize}

\end{document}