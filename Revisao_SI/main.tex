%O documentclass abaixo foi uma ajuda do Alexandre
\documentclass[
	12pt,				% tamanho da fonte
	%openright,			% capítulos começam em pág ímpar (insere página vazia caso preciso)
	%twoside,			% para impressão em recto e verso. Oposto a oneside
	openany,			%Para nao pular folhas quando um paragrafo novo começa. Oposto de Twoside e openright
	a4paper,			% tamanho do papel.
	chapter=TITLE,		% títulos de capítulos convertidos em letras maiúsculas
	section=TITLE,		% títulos de seções convertidos em letras maiúsculas
	%subsection=TITLE,	% títulos de subseções convertidos em letras maiúsculas
	%subsubsection=TITLE,% títulos de subsubseções convertidos em letras maiúsculas
	english,
	brazil				% o último idioma é o principal do documento
]{abntex2}
\usepackage[brazil]{babel}
\usepackage[utf8]{inputenc} %Pacote de linguas
\usepackage[normalem]{ulem}
\usepackage[T1]{fontenc}
\usepackage{lipsum}
\usepackage{cmap}
\usepackage{graphicx}
\usepackage[brazilian,hyperpageref]{backref}
\usepackage[alf]{abntex2cite} % Citações padrão ABNT
\usepackage{rotating}
\usepackage{float}
%%%%%%%%%%%%%%%%%%%%%%%%%%%%%%%%%%
%Comando da image By Filipe

\newcommand{\imagem}[3]{
	\begin{figure}[htb]
		\begin{center}
			\includegraphics[scale=0.5]{#1}
		\end{center}
		\caption{#2}%\label{#3}
	\end{figure}
}

%%%%%%%%%%%%%%%%%%%%%%%%%%%%%%%%%%


%%%%%%%%%%%%%%%%%%%%%%%%%%%%%%%%%%
\autor{2 semestre} % (\\) --> Pula linha

\setlength{\parskip}{0.6cm}
% Identação e espaçamento entre letras
\setlength{\parindent}{1.3cm}
\frenchspacing

% Adicionando idioma
\selectlanguage{brazil}

\titulo{Sistemas de informação\\Revisão}
\data{Abril, 2017}
\tipotrabalho{Trabalho Acadêmico}


\begin{document}

\imprimircapa

\imprimirfolhaderosto

\tableofcontents

\maketitle

\newpage

\chapter{Introdução}

Sistemas de informação estão em todos os lugares, então é necessário que haja um bom entendimento a matéria.

\chapter{Slide 1 - Sistemas de informação (SI)}

%Começo do primeiro slide


Os sistemas de informações tem mudado drasticamente o mundo dos negócios e das pessoas, mas o que são sistemas e sistemas de informação ? 

\textbf{Sistema}
	'Sistema pode ser definido como um conjunto de elementos/componentes que mantêm relações entre si. Os componentes e as relações formam as características específicas do sistema. Para o conjunto de relações entre componentes se associa uma ação/dinâmica e resultados'

\textbf{Sistema de informação}
	'Um Sistema de Informação (SI) é um sistema cujo elemento principal é a informação. Seu objetivo é armazenar, tratar e fornecer informações de tal modo a apoiar as funções ou processos de uma organização'

Um sistema de informação é composto de um sub-sistema social e outro sub-sistema automatizado, onde o social trata das pessoas e dos processos, e o automatizado sobre os computadores e redes de telecomunicação que interligam os elementos sociais.

\textbf{Atividades com informações que contém sistemas de informação}
	\begin{itemize}
		\item Entrada;
		\item Processamento;
		\item Saída;
		\item \textit{Feedback};
		\item A ação de clientes e acionsitas também interage com o sistema de informação da empresa.
	\end{itemize}

\textbf{Os sitemas de informação vão muito aĺém dos computadores}

Os sistemas de informação conseguem abrangir muito mais do que apenas computadores, eles conseguem unificar a organização, os processos e pessoas, e os computadores para que consiga oferecer solução para os mais diversos problemas da organização.
Eles se divedem nos tópicos de:
	\begin{itemize}
		\item Organizacional: Encontrasse processos, as formas hierarquicas da empresa, e quando a empresa utiliza os sitemas de informação, a cultura organizacional desta reflete nos sitemas de informação. Veja a \textit{United Parcel Service} (UPS) que se preocupa muito com o cliente e a forma de atendimento;
		\item Pessoa: As pessoas são quem desenvolvem, mantém e utilizam o sistema de informação, a ação de cada pessoa sobe os sistemas de informação podem afeta-lo de forma positiva ou negativa;
		\item Tecnologia: São as plataformas ais quais os sistemas de informação são construidos sobre. Sendo \textit{Software}, \textit{Hardware}, gerenciamento de dados, redes e telecomunicação.
	\end{itemize}

Porém há problemas que são enfrentados por cada um desses itens:
	\begin{itemize}
		\item Organizacional:
			\begin{itemize}
				\item Processos ultrapassados;
				\item Cultura pouco colaborativa;
				\item Complexidade das tarefas;
				\item Recursos inadequados ou ineficientes.
			\end{itemize}
		\item Tecnológico:
		\begin{itemize}
			\item \textit{Hardware} antigo;
			\item Capacidade inadequada do banco de dados;
			\item Mudanças tecnológicas muito rápida.
		\end{itemize}
		\item Pessoas
			\begin{itemize}
				\item Falta de trenamento;
				\item Falta de participação;
				\item Administração ineficiente.
			\end{ìtemize}
	\end{itemize}

A solução para todos estes problemas é um processo continuo.

Veja que após a implantação do sistema é apenas o começo de uma nova etapa, nesta nova etapa, é feito a verificação do que está ou não funcionando e em seguida análisar o que pode e o que não pode ser feito para a solução do problema. 

Todos estes passos se traduzem em quatro passos:

\begin{itemize}
	\item \textbf{1°} Identificação do problema;
	\item \textbf{2°} Propostas de solução; 
	\item \textbf{3°} Avaliação das propostas e escolhas da solução;
	\item \textbf{4°} Implantação.
\end{itemize}

Com estes passos é possível realizar a criação de soluções consistentes para os mais diversos problemas, perceba que, quando a industria tem problemas que não são resolvidos facilmente, estes se tornam desafios, e os sistemas de informação ajudam total ou parcialmente na solução destes desafios.

\subsection{O papel dos sistemas de informação no ambiente de negócios contemporâneo}

Cada vez mais os sistemas de informação vem tomando conta dos ambientes corporativos, já é extremamente comum ver empresas com sites e blogs para postar conteúdo com relação a seus serviços e desta forma disseminar sua marca. O \textit{marketing} digital já se tornou algo poderoso e é muito utilizado por todas as empresas. Para perceber isso basta entender que o famoso \textit{spam} foi modificado, estruturado e tornado uma grande ferramentas de \textit{marketing} com os \textit{e-mail marketing}. Até mesmos leis foram criadas para assegurar que dados e informações se tornassem algo a se preocupar por parte das empresas que oferecem serviços \textit{on-line}.

\subsubsection{As novas tendências dos sistemas de informação}

%https://antonioricardo.org/2013/03/28/o-que-e-saas-iaas-e-paas-em-cloud-computing-conceitos-basicos

Como dito anteriormente, os SI crescem a cada dia, é possível perceber isso com o avanço de técnologias como:
	\begin{itemize}
		\item Computação em nuvem;
		\item Software como serviço (SaaS - \textit{Software as a service});
		\item Infraestrutura como serviço (IaaS - \textit{Infraestruct as a service});
		\item Plataforma como serviço (PaaS - \textit{Plataform as a service});
		\item Trabalho remoto amplamente utilizado;
		\item Cada vez mais os serviços remotos diminuem os valores das operações.
	\end{itemize}

A evolução das técnologias trouxeram também novas formas de negócios, hoje existem trabalhos pare resolver problemas que a 5 anos atrás não existiam. Com esta evolução as empresas perceberam que seria necessário investir em tecnologia, para se manterem competitivas e assim se fez, hoje as empresas investem em T.I para conseguir:
\begin{itemize}
	\item Excelência operacional
		\begin{itemize}
			\item A excelência operacional gera mais lucros, veja que há empresas que são gigantes por terem técnologias que as auxiliam nisto, como exemplo a Amazon, empresa de varejo gigantesca que oferece os mais variados serviços. E com sistema de informação consegue realizar entregas de seus pedidos em até alguns minutos depois que foram feitos.
		\end{itemize}
	\item Novos produtos, serviços e modelos de negócios
		\begin{itemize}
			\item As novas técnologias permitem a criação de novas formas de negócios, como citado anteriormente, veja a indústria fonográfica que a alguns anos atrás era gigantesca e hoje perde cada vez mais espaço para empresas como o \textit{Spotify} que permitem que qualquer artista publique suas obras sem terem grandes fortunas.
		\end{itemize}
	\item Relacionamento mais estreito com clientes e fornecedores
		\begin{itemize}
			\item Com a melhora e velocidade dos atendimento subindo, os clientes sempre retornam.
		\end{itemize}
	\item Melhor tomada de decisão
		\begin{itemize}
			\item Sistemas como o \textit{Business Intelligence} ou BI conseguem interpretar as informações de todas as operações realizadas pela empresa, e emitir relatórios precisos sobre toda a empresa, permitindo uma tomada de decisão muito melhor.
		\end{itemize}
	\item Vantagem competitiva
		\begin{itemize}
			\item Dependendo do ramo em que a empresa atua, a técnologia pode fazer com que o produto oferecido pode ser superior ao concorrente e ainda mais barrato.
		\end{itemize}
	\item Sobrevivência
		\begin{itemize}
			\item Mesmo se a empresa não trabalha com técnologia ela será obrigada comprar técnologia para se manter ativa em seu ramo.
		\end{itemize}
\end{itemize}


Assim é possível perceber que as empresas investem em técnologia para ter:
	\begin{itemize}
		\item Vantagens competitivas,
		\item Melhores serviços,
		\item Menos erros,
		\item Maior precisão, 
		\item Produtos de melhor qualidade,
		\item Aperfeiçoamento,
		\item Melhor eficiência,
		\item Maior produtividade, 
		\item Maiores oportunidades,
		\item Administração mais eficiente,
		\item Automatização de tarefas rotineiras,
		\item Custos reduzidos,
		\item Maior e melhor controle sobre as operações,
		\item Melhores tomadas de decisões..
	\end{itemize}

\subsection{Como o SI afeta as carreiras}
Como citado os sistemas de informação tornaram todas as operações mais dinâmicas, e até mesmos as profissões tiveram de se adequar a este novo meio de trabalho. Perceba que hoje muitas profissões não realizam suas atividades sem o SI. Por exemplo o \textit{Marketing} tem seu trabalho totalmente dependete dos sistemas de informação, ou mesmo profissões mais antigas como a administração, que para gerir empresas gigantescas utilizam fortimente dos sistemas.

%Fim do primeiro slide

%======
%Começo do segundo slide

%NATUREZA DO CONHECIMENTO
%CLASSIFICAÇÃO DOS SISTEMAS DE INFORMAÇÃO
%SI NA NOVA EMPRESA DO SÉCULO 21

\chapter{Slide 2 - Classificação dos sistemas de informação}

\textbf{Dado X Informação X Conhecimento}
	\begin{itemize}
		\item Dados: Elemento puro, vindo de um certo evento;
		\item Informação: É o dado analisado e contextualizado, este também vem de um conjunto de dados;
		\item Conhecimento: São as informações processadas e compreendidas, este cria sistemas mentais que permitem seu detentor de criar e tomar decisões.
	\end{itemize}

Os conhecimentos ainda são divididos em dois:
	\begin{itemize}
		\item Tácito:
			\begin{itemize} 
				\item Percepções;
				\item Ideias;
				\item Experiências.
			\end{itemize}
		\item Explícito:
			\begin{itemize} 
				\item Melhores Práticas;
				\item Políticas;
				\item Procedimento;
				\item Informações;
				\item Documentos.
			\end{itemize}
\newpage
Em resumo:
\imagem{tab_con.png}{Dado, informação e conhecimento}{rótulo_para_referência}

\subsection{Classificação dos sistemas de informação}

	\begin{itemize}
	\item[] \textbf{A} – Conforme as comunidades de clientes predominantes
	Esta é uma das formas mais interessantes de se classificar os sistemas de informação, já que permitem uma maior proximidade ao problema dos clientes, este tipo de sistema consegue ser:
		\begin{itemize}
			\item Estrategico: Por permitir planejamento de ações a longo prazo;
			\item Tático: Suportam supervisão, controle e tomada de decisão e atividades administrativas;
			\item Operacional: Sistemas que supervicionam as atividades elementares as transações na organização;
		\end{itemize}
	\item[] \textbf{B} – Conforme as tecnologias empregadas
	Sistemas que são classificados de acordo com as técologias utilizadas e suas foras de trabalho:
		\begin{itemize}
			\item Sistemas de processamento em lote, ou sistemas “batch”
				\begin{itemize}
					\item É caracterizado pela forma de programação estruturada, com armazenamento de dados hierárquico, interação apenas com o operador técnico, utilizado de fitas magnéticas e cartões perfurados. Por fim é possível perceber que este é um formato mais antigo de técnologia;
				\end{itemize}
			\item Sistemas de transação com o usuário, ou sistemas “on-line”
				\begin{itemize}
					\item Programação estruturada;
					\item Projeto e análise estruturada;
					\item Armazenamento de dados em rede;
					\item Interação direta com o usuário final;
					\item Sua forma padronizada de projeto aumentou a qualidade dos serviços e sua velocidade.
				\end{itemize}
			\item Sistemas cliente servidor, ou sistema “client-server”
				\begin{itemize}
					\item Sistema de armazenamento, processamento e impressão de dado de dados se tornou distribuido, aumentando a flexibilidade dos projetos;
					\item A flexibilidade trouxe a possíbilidade de projetos específicos para a área de atuação de cada empresa;
					\item Os sistemas começaram a ser oferecidos por processos, como exemplo há os ERPs, que contem diversos processos dentro de um único sistema para atender a publicos específicos.
				\end{itemize}
			\item Sistemas baseados em Internet, ou solução “e-business”
				\begin{itemize}
					\item Utilizam a internet para realizar a comunicação com o cliente;
					\item Facilita o oferecimento de produtos ao cliente;
					\item Comunicação rápida e fácil;
					\item Atinge os mais diversos clientes.
				\end{itemize}
		\end{itemize}
	\item[] \textbf{C} – Conforme os processos de negócio atendidos
	Estes são sistemas que tem funcionalidades para públicos específicos e seus negócios. Por exemplo um setor administrativo conta com o ERP, para gerir suas funções com mais facílidade e velocídade.
	Já a área de comunicação com o cliente utiliza de sistemas CRM, que facilitam e estruturam seu trabalho.
	\end{itemize}

	%http://www.inf.ufrgs.br/~vrqleithardt/Teaching/AULA%20SEMANA%202/Conceitos%20b%E1sicos%20em%20CRM.pdf
	\newpage
	\subsection{ERM – ENTERPRISE RELATIONSHIP MANAGEMENT}
	Com o passar do tempo as empresas começaram a buscar melhores formas de trabalho, para aumentar seus lucros e produtividade, essas novas formas de trabalho sem denominam ERM.
	Esse se divide em três grandes grupos:
	\begin{itemize}
		\item Front-office: Cuida da 'vitrine' da empresa, a forma com que o cliente será tratado, este grupo engloba outros conceitos e técnologias como o CRM. Tem por objetivo melhorar as vendas e o \textit{marketing} da empresa. O front-office abrange:
			\begin{itemize}
				\item CRM;
				\item \textit{e-commerce (B2C)};
				\item SAF (Sales Force Automation);	
			\end{itemize}
		\item Back-office: O Back-office conta com toda a estrutura necessária para a realização das atividades da empresa, seja atráves de sistemas como O ERP ou com uma boa administração. Ou seja tudo que será base para a realização dos processos da empresa se encaixam aqui, o back-office abrange:
			\begin{itemize}
				\item PRM;
				\item \textit{e-commerce (B2B)};
				\item ERP/SCM;
			\end{itemize}
		\item Bussines Intelligence: Aqui são realizados as ánalises das informações mantidas pela empresa para ajuda-la a entender sua cituação atual e tomar a melhor decisão possível. Este abrange:
			\begin{itemize} 
			\item Relatórios (EIS);
			\item Dados analiticos;
			\item Padrões (\textit{Data mining}).
	\end{itemize}

	\subsection{Infra-estrutura de TI}

	A infra-esturtura é parte vital para o funcionamento de qualquer sistemas de informação, já que é ela que oferece todos os recursos necessários para a implantação e mantimento de todos os demais sistemas. Aqui discutirei apenas a infra-estrutura necessária para o mantimento de um ERP.

	\subsubsection{ERP - Sistemas de Gestão Integrada} 
	O ERP é um sistema que integra todos os dados e processos da empresa em um único sistema. Esta integração é total, ele únifica todos os processos da empresa, facilitando o acesso a informação.
	Para o ERP ser considerado um Sistema de informação ele devem:
	\begin{itemize}
		\item Oferecer qualquer informação da empresa em um único \textit{software};
		\item Todos os modulos utilizados devem ser iguais, alterando apenas suas funções
	\end{itemize}
	O ERP trás algumas vantagens para a empresa que o utiliza:
	\begin{itemize}
		\item Otimiza o tempo de trabalho;
		\item Reduz custos;
		\item Melhora o processo de tomada de decisão;
		\item Elimina atividades redundantes.
	\end{itemize}
	\\Desvantagens:
	\begin{itemize}
		\item Nem sempre trabalha com custo X benefício;
		\item O ato de implanta-lo e utilizar não signigica melhora na empresa;
		\item Dependência dos fornecedores de pacotes.
	\end{itemize}

	Fatores que fazem um ERP ser bem utilizado na empresa:
	\begin{itemize}
		\item Planejamento adequado;
		\item Expectativas realistas;
		\item Apoio da direção;
		\item Comprometimento;
		\item Equipe competente.
	\end{itemize}

	Por fim, o ERP é um \textit{software} que integra todas a empresa e ajuda a manter todas as informações centralizadas e organizadas, é separado por módulos, porém tem uma interface padrão que ajuda o usuário a entender melhor e com pouco tempo de uso não ter mais problemas para realizar as atividades dentro do sistema.
\end{document}