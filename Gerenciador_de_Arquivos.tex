\documentclass[12pt,a4paper,openany,oneside]{abntex2}
\usepackage[brazil]{babel}
\usepackage[utf8]{inputenc}
\usepackage[normalem]{ulem}
\usepackage[T1]{fontenc}\usepackage{lipsum}
\usepackage{cmap}
\usepackage{graphicx}


% % % % % % % % % % % % % % % % % % % % % % % % % % % % % % % % % % % % % % % % % % % %
\titulo{Gerenciador de arquivos}
\autor{Felipe Menino\\Gabriel Takarashi\\Kelvin Severino\\Glaucio de Melo\\Rodrigo Pavret}
\local{São josé dos Campos}
\data{Novembro,2016}
\orientador{Antonio Egydio Santiago Graça}
\tipotrabalho{monografia}
% % % % % % % % % % % % % % % % % % % % % % % % % % % % % % % % % % % % % % % % % % % %
\begin{document}

\imprimircapa

\imprimirfolhaderosto

\tableofcontents

\maketitle

\section{Introdução}



\section{História}

O gerenciador de arquivo nasceu de programas especificos para o sistema IBM em 1984, esses eram utilizados como \textit{Shell} nesses sistemas. Foi um enorme avanço para as áreas de técnologia da informação, já que com o gerenciamento de arquivo, a capacidade de armazenar e tratar muitos arquivos cresceu de forma muito rápida. As primeiras versões eram utilizadas em computadores especificos da IBM, porém em 1986 surgiu o Norton Commander, que acabou sendo adaptado para muitos outros sistemas.

\section{Arquivos e diretorios}

Antes de se entrar nos gerenciadores de arquivos é necessário entender como o sistema operacional trata e armazena arquivos e diretórios, já que é através deste principio que o gerenciador de arquivos trabalha, assim o gerenciador de arquivo utiliza as funções proprias e do sistema de arquivos para realizar tarefas como a criação, exclusão, copia e renomeio dos diretorios e arquivos.

\subsection{Arquivos}

Os computadores são capazes de armazenar grandes quantidades de informação, todas essas em formato de arquivos.

 \begin{quote}
"Um arquivo é uma sequência de informações binárias, ou seja, uma sequência de 0 e 1. Este arquivo pode ser armazenado para guardar um vestígio destas informações.", Kyoskea
\end{quote}

Os arquivos tem algumas caracteristicas que os definem, e são elas:

\begin{itemize}

\item Nome: Este é como o rótulo do arquivo, ou seja, é utilizado para que o usuário encontre o arquivo correto;
\item Tipo: O tipo do arquivo está relacionado a que tipo de informação ele carrega, essa podendo ser uma música, video ou foto;
\item Extensão: A extensão do arquivo é utilizada para que o S.O consiga identificar qual é o tipo do arquivo e qual programa se adequa para a execução daquele arquivo. É importante lembrar que a extensão do arquivo não interfere em seu tipo, este como já citado é apenas uma informação para que o sistema operacional identifique qual aplicativo se adequa a aquele arquivo, mas também não é um fator determinante para a execução do mesmo.

\end{itemize}

Existem ainda as funções que podem ser realizadas nos arquivos, essas são empregadas nas mais diversas atividades.

\begin{itemize}

\item Criação: Realiza a criação de um arquivo vazio;
\item Exclusão: Remove o arquivo do espaço da memoria onde estava alocado;
\item Abertura: Aloca na memoria os atributos necessário para a execução do arquivo;
\item Fechamento: Remove da memoria os atributos alocados anteriormente na abertura;
\item Leitura: Abre o arquivo apenas para leitura de seu conteúdo;
\item Adição: Operação que permite apenas adicionar informações ao final do arquivo;
\item Busca: Faz a leitura completa do arquivo, permitindo a pesquisa no mesmo;
\item Ver atributos: Este faz a verificação dos atributos que foram concedidos para aquele arquivo, como permissões, dono entre outros;
\item Renomear: Faz a renomeação do rotulo do arquivo, ou seja, muda seu nome.

\end{itemize}

\subsubsection{Alocação de arquivos}

 Para que o sistema operacional realize o armazenamento de qualquer arquivo é necessário que ele saiba os blocos que estão livres para armazenamento, e ainda os blocos que estão ocupados e quais arquivos estão ocupando certo espaço. Assim, para que esta relação seja feita são utilizadas as mais diversas formas de gerenciamento.

\begin{itemize}

\item Alocação contígua

	\begin{itemize}
		\item Este é o sistema mais simples para a alocação de arquivos em disco, e era utilizado nos primeiros sistemas operacionais. A técnica utilizada era bastante simples e consistia em armazenar arquivos em conjuntos de blocos contíguos (Blocos vizinhos), se destacava por ser bastante simples e ter um bom desempenho, por outro lado depois de uma certa quantidade de arquivos este sistema começou a aprensetar problemas além de que os arquivos não poderiam crescer muito;
	\end{itemize}

%Vai uma figura aqui <--- alocContig.jpg -->

\item Alocação por lista encadeada
	
	\begin{itemize}
		\item Nesta forma de alocação é utilizada uma lista encadeada para indicar os espaços ocupados em disco pelos arquivos, descartando a utilização do sistema de arquivos contiguos. A primeira palavra de cada bloco é utilizada como um ponteiro para os próximos blocos que irão armazenar as informações do arquivo.Com este sistema é possivel utilizar qualquer bloco disponivel na unidade de armazenamento, não há fragmentação externa e a entrada do díretorio só precisa conter o endereço do primeiro bloco que está armazenando o arquivo. Por outro lado este é um formato de alocação lento, já que sempre ele terá de percorrer toda a lista encadeada para econtrar o arquivo requisitado, sem contar que implementar este método é bastante complicado;
	\end{itemize}

%Aqui vai outra imagem <-- alocEncad.jpg -->

\item Alocação com lista encadeada usando tabela na memória

	\begin{itemize}
		\item Anteriormente foi citado que o problema da alocação por lista encadeada era a maneira com que os ponteiros se organizavam e assim a busca se tornava lenta. Para resolver este problema foi criada uma tabela nomeada FAT (\textit{File Allocation Table}), com este método no momento do acesso aleatório tudo se torna mais rápido, já que todo o esquema de ponteiros estará alocado na memória. Com todo o sistema de ponteiros na memória surgiu outro problema, a quantidade de memória utilizada para armazenar toda esta estrutura.;
	\end{itemize}

\item Alocação com lista usando um índice

	\begin{itemize}
		\item Para solucionar todos os problemas causados até então pelo sistema de alocação com listas, foi criado uma nova técnica. Esta agora criava um indíce para cada arquivo e os blocos que compoem este, e este indíce fica alocado em outro bloco. O diretório possui um ponteiro para o bloco onde está armazenado o indíce do arquivo requisitado. Mesmo utilizando o acesso randômico sua implementação é mais simples que os demais. O único problema que persegue este padrão é a necessidade de utilizar a unidade de armazenamento para alocar as tabelas. O que nos dias atuais não é um problema tão grande como anteriormente;

	\end{itemize}

% Aqui vai outa imagem <-- alocIndice.jpg -->

\item Inode

	\begin{itemize}
		\item Em cada arquivo é armazenado uma pequena tabela, esta contendo todos os atributos e endereços dos blocos em disco deste arquivo. Os primeiros endereços ficam armazenados dentro do próprio arquivo, e caso esse seja pequeno todas as informações ficam armazenados no mesmo arquivo, assim o conteúdo do arquivo so é transmitido da memória para o disco quando o mesmo é aberto.
	\end{itemize}



\end{itemize}

Todos estes processos são bastante utilizados pelos gerenciadores de arquivos para a realização das funções requisitadas pelos usuários, já que como será visto a frente, o gerenciador de arquivos apenas geri as funções do sistemas de arquivos.

\subsection{Diretórios}

Os diretorios também são parte vital para o entendimento do gerenciador de arquivos, isto porque é ele o responsavel pela organização de todos os arquivos. Veja que um diretorio é uma subdivisão lógica do sistema de arquivos, e seu intuito é  realizar a junção e organização dos arquivos.
Os diretorios são organizados pelo sistema operacional de maneira hierarquica, assim todos os diretorios ficam alocados em outros diretorios, estes separados por usuários para que assim possam ser recuperados facilmente.

\section{Gerenciador de arquivos}

O gerenciador de arquivo é uma das partes vitais para o funcionamento do sistema operacional, ele é o responsavel pela criação, leitura, escrita e exclusão dos arquivos dentro da unidade de armazenamento.

\subsection{Funcionamento}

O gerenciador de arquivos trabalha juntamente com o sistema de arquivos, ele é o responsavel em gerenciar todas as ações que dizem respeito aos diretorios e arquivos presentes na unidade de disco. Ele trabalha diretamente com o sistema de arquivos, assim ele pega os comandos enviados pelo usuário e inerpreta enviado para o sistema de arquivos. Outro fato interessante é que boa parte das funções do gerenciador de arquivos está diretamente relacionada os sistema de arquivos utilizado.
Todo o sistema de hierarquia das pastas e arquivos e gerido pelo gerenciador de arquivos, e ainda quando há indexação de arquivos por parte do sistema de arquivos, o gerenciador deve cuidar e manter a organização durante este processo, já que tudo gerando neste momento será utilzado por ele para realizar buscas nas arvores de arquivos do sistema operacional.

%Imagem do funcionamento

\subsection{Tipos de gerenciadores de arquivos}

Os gerenciados de arquivos tem diversos tipos esses mudando ao longo dos tempos, permitindo que cada vez mais arquivos fossem empregados a unidade de disco.

%Ainda terá mais conteúdo

\subsubsection{\textit{File-list file manager}}

Esta forma de gerenciador de arquivos é menos conhecida e mais antiga do que os gerenciados OFM  (Tratados mais a frente). Por ser bastante antigo aprensenta muitas limitações, não tendo muitas funções. Porém no periodo em que foi desenvolvido apresentava uma caracteristica extremamente interessante, este conseguia classificar o arquivo com qualquer atributo presente neste.

\subsubsection{\textit{Orthodox File Managers} (OFM)}


Gerenciadores de arquivos ortodoxos (OFM) são gerenciadores de arquivos de textos com base em menus. Os gerenciadores OFM são a maior familha de gerenciadores de arquivos. Sua simplicidade permite aos desenvolvedores criar ferramentas e aplicações que utilizem deste gerenciador sem altera-lo. 
Este é um padrão de gerenciador de arquivos muito utilizado em servidores por dar ao \textit{SYSADMIN} uma visão ampla do ambiente em que ele está trabalhando

Abaixo são listados as propriedades que fazem dos OFM se tornarem gerenciadores de arquivos extremamente uteis para todo tipo de usuário

\begin{itemize}
	\item Propriedades: \\
	\begin{itemize}
		\item É extremamente extavel e flexivel;
		\item Faz a integração do sistema gerenciador de arquivos com o Shell do sistema operacional;
		\item O terminal é iniciado normalmente, mas pode introduzir a as janelas a qualquer momento.
	\end{itemize}

Agora que já se entende um pouco das propriedades empregadas nesses gerenciadores veja caracteristicas que acompanham todos os gerenciadores de arquivos desta familia

	\item Caracteristicas: \\ 
	\begin{itemize}
		\item São compostos por três janelas, duas dessas chamadas de painéis, que contem a interação com o sistema de arquivos, e a terceira é a interface gráfica. Os arquivos podem ser movidos de um painel para o outro, e esta é uma das caracteristicas que torna os OFM muito util para \textit{SYSADMINS};
		\item A forma e atributos que estão sendo exibidos dos arquivos podem ser alterados da maneira com que o usuário desejar;
		\item Uso de atalhos de teclado externos;
		\item O gerenciador de arquivos OFM tira a necessidade da utilização do Mouse;
		\item Toda a ajuda que o usuário necessita pode ser encontrada no manual de usuário;
		\item Os tipos mais básicos de arquivos, como os de texto, tem visualização no momento em que é selecionado;
		\item Interfaces com guias (Utilizado quando a GUI utiliza o OFM para gerenciar os arquivos)
	\end{itemize}

Todas essas caracteristicas tornam o gerenciador de arquivos ortodoxo bastante conhecido e utilizado por todo o tipo de usuário, como já citado anteriormente.
\end{itemize}

%Colocar uma explicação para cada uma das imagens
%Colocar imagens de exemplo com o Gnome Commander, DuosCommander e o Norton Commander


\subsection{\textit{Navigation file manager}}

Os gerenciadores de arquivos de navegação (\textit{Navigation file manager}), surgiram com o advento das interfaces gráficas nos dispositivos. Este tipo de gerenciador de arquivos apresenta uma arvore de todos os arquivos do sistema operacional.

\begin{itemize}
	\item Caracteristicas
	\begin{itemize}
	\item Possui dois paineis, um com a estrutura arvore dos arquivos do sistema, e outro com o conteúdo de cada um dos itens selecionados no primeiro painel;
	\item Tem similaridades a um navegador de \textit{internet}, já que permite a navegação por todos os arquivos presentes no computador;
	\item Todas as ações de mover os arquivos são baseadas em metaforas de arraste e solte. Para isso é utilizado o sistema de área de transferência, onde o arquivo ao ser copiado, fica nesta área de transferência, até ser remanejado até outro diretório.
	\end{itemize}
\end{itemize}

\subsection{\textit{Spatial file manager}}
%\subsection{\texit{Spatial file manager}}

A forma de arquivos espaciais é um formato de gerenciador de arquivos que buscou tratar os arquivos de maneira mais "real", assim cada arquivo era tratado como um objeto físico, e o gerenciador de arquivos tenta tratar estes como se fosse uma pessoa, assim ao abrir uma pasta é criado uma nova janela, para que a ideia de uma coisa por vez seja efetiva neste formato. Foi utilizado durante um curto periodo de tempo.

\subsection{Gerenciador de arquivos 3D}

Os arquivos 3D são ainda bastante utilizados, estes trazem uma visão mais real dos arquivos, o que permite a adminsitradores de redes e banco de dados, perceber o espaço que certos arquivos estão ocupando em seu sistema, isso ajuda também na melhoria continua dos sistemas e pode ser utilizado junto a outros  gerenciadores de arquivos, servindo até mesmo como uma ferramenta dentro do gerenciador de arquivos do sistema.

\end{document}