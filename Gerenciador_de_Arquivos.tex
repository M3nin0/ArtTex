\documentclass{article}
\usepackage[utf8]{inputenc}
\usepackage{graphicx}

\title{Gerenciador de arquivos}
\author{Felipe}
\date{Novembro, 2016}

\begin{document}

\maketitle

\section{Introdução}



\section{Gerenciador de arquivos}

\subsection{Arquivos}

Os computadores são capazes de armazenar grandes quantidades de informação, todas essas em formato de arquivos.

 \begin{quote}
"Um arquivo é uma sequência de informações binárias, ou seja, uma sequência de 0 e 1. Este arquivo pode ser armazenado para guardar um vestígio destas informações.", Kyoskea
\end{quote}

Os arquivos tem algumas caracteristicas que os definem, e são elas:

\begin{itemize}

\item Nome: Este é como o rótulo do arquivo, ou seja, é utilizado para que o usuário encontre o arquivo correto;
\item Tipo: O tipo do arquivo está relacionado a que tipo de informação ele carrega, essa podendo ser uma música, video ou foto;
\item Extensão: A extensão do arquivo é utilizada para que o S.O consiga identificar qual é o tipo do arquivo e qual programa se adequa para a execução daquele arquivo. É importante lembrar que a extensão do arquivo não interfere em seu tipo, este como já citado é apenas uma informação para que o sistema operacional identifique qual aplicativo se adequa a aquele arquivo, mas também não é um fator determinante para a execução do mesmo.

\end{itemize}

Existem ainda as funções que podem ser realizadas nos arquivos, essas são empregadas nas mais diversas atividades.

\begin{itemize}

\item Criação: Realiza a criação de um arquivo vazio;
\item Exclusão: Remove o arquivo do espaço da memoria onde estava alocado;
\item Abertura: Aloca na memoria os atributos necessário para a execução do arquivo;
\item Fechamento: Remove da memoria os atributos alocados anteriormente na abertura;
\item Leitura: Abre o arquivo apenas para leitura de seu conteúdo;
\item Adição: Operação que permite apenas adicionar informações ao final do arquivo;
\item Busca: Faz a leitura completa do arquivo, permitindo a pesquisa no mesmo;
\item Ver atributos: Este faz a verificação dos atributos que foram concedidos para aquele arquivo, como permissões, dono entre outros;
\item Renomear: Faz a renomeação do rotulo do arquivo, ou seja, muda seu nome.

\end{itemize}

\subsubsection{Alocação de arquivos}
 

\subsection{Diretórios}

Os diretorios também são parte vital para o entendimento do gerenciador de arquivos, isto porque é ele o responsavel pela organização de todos os arquivos. Veja que um diretorio é uma subdivisão lógica do sistema de arquivos, e seu intuito é  realizar a junção e organização dos arquivos.
Os diretorios são organizados pelo sistema operacional de maneira hierarquica, assim todos os diretorios ficam alocados em outros diretorios, estes separados por usuários para que assim possam ser recuperados facilmente.


\end{document}