\documentclass{article}
\usepackage[utf8]{inputenc}

\title{Gerenciador de arquivos}
\author{Felipe}
\date{Novembro, 2016}

\begin{document}

\maketitle

\section{Introdução}



\section{Gerenciador de arquivos}

\subsection{Arquivos}

Os computadores são capazes de armazenar grandes quantidades de informação, todas essas em formato de arquivos.

 \begin{quote}
"Um arquivo é uma sequência de informações binárias, ou seja, uma sequência de 0 e 1. Este arquivo pode ser armazenado para guardar um vestígio destas informações.", Kyoskea
\end{quote}

Os arquivos tem algumas caracteristicas que os definem, e são elas:

\begin{itemize}

\item Nome: Este é como o rótulo do arquivo, ou seja, é utilizado para que o usuário encontre o arquivo correto;
\item Tipo: O tipo do arquivo está relacionado a que tipo de informação ele carrega, essa podendo ser uma música, video ou foto;
\item Extensão: A extensão do arquivo é utilizada para que o S.O consiga identificar qual é o tipo do arquivo e qual programa se adequa para a execução daquele arquivo. É importante lembrar que a extensão do arquivo não interfere em seu tipo, este como já citado é apenas uma informação para que o sistema operacional identifique qual aplicativo se adequa a aquele arquivo, mas também não é um fator determinante para a execução do mesmo.


\end{itemize}

\end{document}